Section 1:

a. false (asymptotic profile of tangent makes upper bounds impossible)
b. false (e^n outgrows 2^n even when 2^n multiplied by a constant)
c. true (c=2, n_0=1)
d. true (c=\frac{1}{9}, n_0=1)
e. true (c_1=\frac{1}{15}, c_2=1, n_0=1)
f. true (c=1, n_0=1, use definition of positive function, prove for cases g(n) > f(n) and f(n) > g(n))
g. true (c_1=1, c_2=2, for proving big-Oh use c_2=2 and subtract both sides by f, for big-omega use c_1=1 and subtract both sides by f)
h. false (simplify to g(n) = n^{\frac{5}{2}}, n^{\frac{5}{2}} <= cn^{2} fails for all constants c)











Section 2:

a. T(n) = 5T(n - 3)
unrolling yields one term which is 5^{\frac{n-5}{3}}*c. Can then prove that T(n) = \Theta (5^{\frac{n}{3}})

b. (use master theorem II from lecture slides)

c. T(n) = 6T(\frac{n}{4}) + n
Master theorem: a=6, b=4, c=1, k=1   \alpha = 1.5 > 1, big-theta(n^{log_{4}(6)})

d. T(n) = T(n - 2) + 10
unrolling, \frac{n}{2} terms each of which is at most 10, so 10*\frac{n}{2}, so big-theta(n) (linear)











Section 3:

3.a:
Pf by Contradiction
Suppose FTSOC that no more than ceiling(157/12) - 1 = 13 students have 
birthdays in the same month.

The total number of students is at most 12(ceiling(157/12)-1) < 12((157/12 + 1) - 1) = 157.
This is a contradiction because there are 157 students.

3.b:
Pf by Induction
For n = 1, SUM{1,1}(2i - 1) = (2(1) - 1) = (2 - 1) = 1 = 1^2
Assume that for some n = k, SUM{1,k}(2i - 1) = (2(k) - 1) + (2(k - 1) - 1)
+ ... + (2(1) - 1) = k^2

For n = k + 1, SUM{1,k + 1}(2i - 1) = (2(k + 1) - 1) + (2(k) - 1) + ... + 
(2(1) - 1) = (2(k + 1) - 1) + SUM{1,k}(2i - 1) = (2(k + 1) - 1) + k^2
= (2k + 2 - 1) + k^2
= (2k + 1) + k^2
= k^2 + 2k + 1
= (k + 1)^2. QED


3.c
Combination problem
Want number of ways to complete the wanted outcome, divided by the number of total possible ways to complete any outcomes.
(20C3 * 12C6) / 32C9 = ~3.755%
